% !TeX program = LuaLaTeX

\documentclass[tikz]{standalone}


\usepackage{amsmath}
\usepackage{pgfplots,pgfplotstable}

\pgfplotsset{compat=1.7}
\pgfplotstableset{format=inline, row sep=crcr}


\newcommand*\measurement[2]{%
  \edef\next{\noexpand
    \pgfplotstableread{\unexpanded{#2}}{\expandafter\noexpand
      \csname measurement#1\endcsname}
  }\next
}


\input{\filename}


\begin{document}

\begin{tikzpicture}
\begin{axis}[
  xlabel={\# of packets}, ylabel={Occurrences},
  xmin=0, ymin=0, xmax=\trainlength
]
\begingroup\edef\next{\endgroup\noexpand
  \addplot [ybar interval] table from \expandafter\noexpand\csname
    measurementdistribution-plateau-lengths\endcsname;
}\next
\end{axis}
\end{tikzpicture}

\begin{tikzpicture}
\begin{axis}[
  xlabel={\# of packets}, ylabel={CCDF $P(X>x)$},
  xmin=0, ymin=0, xmax=\trainlength
]
\begingroup\edef\next{\endgroup\noexpand
  \addplot [draw=blue] table from \expandafter\noexpand\csname
   measurementccdf-plateau-lengths\endcsname;
}\next
\end{axis}
\end{tikzpicture}

\begin{tikzpicture}
\begin{axis}[
  xlabel={Duration (\mu s)}, ylabel={Occurrences},
  xmin=0, ymin=0, xmax=40000
]
\begingroup\edef\next{\endgroup\noexpand
  \addplot [ybar interval] table from \expandafter\noexpand\csname
    measurementdistribution-jump-distances\endcsname;
}\next
\end{axis}
\end{tikzpicture}

\begin{tikzpicture}
\begin{axis}[
  xlabel={Duration (\mu s)}, ylabel={CCDF $P(X>x)$},
  xmin=0, ymin=0, xmax=40000
]
\begingroup\edef\next{\endgroup\noexpand
  \addplot [draw=blue] table from \expandafter\noexpand\csname
    measurementccdf-jump-distances\endcsname;
}\next
\end{axis}
\end{tikzpicture}

\begin{tikzpicture}
\begin{axis}[
  xlabel={\# of packets}, ylabel={CCDF $P(X>x)$},
  xmin=0, ymin=0, xmax=\trainlength
]
\begingroup\edef\next{\endgroup\noexpand
  \addplot [draw=blue] table from \expandafter\noexpand\csname
    measurementccdf-plateau-lengths\endcsname;
}\next
\draw [red, dashed] (axis cs:0,\ccdfplateaufourthy) -- (axis cs:40,\ccdfplateaufourthy)
  node[inner sep=1pt, fill] {} -- (axis cs:40,0);
\end{axis}
\end{tikzpicture}

\end{document}
